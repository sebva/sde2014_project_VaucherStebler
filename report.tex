\documentclass{paper}

%\usepackage{times}
\usepackage{epsfig}
\usepackage{graphicx}
\usepackage{amsmath}
\usepackage{amssymb}
\usepackage{color}
\usepackage{url}
\graphicspath{ {images/} }
\usepackage{float}


% load package with ``framed'' and ``numbered'' option.
%\usepackage[framed,numbered,autolinebreaks,useliterate]{mcode}

% something NOT relevant to the usage of the package.
\setlength{\parindent}{0pt}
\setlength{\parskip}{18pt}






\usepackage[latin1]{inputenc} 
\usepackage[T1]{fontenc} 

\usepackage{listings} 
\lstset{% 
   language=Matlab, 
   basicstyle=\small\ttfamily, 
} 



\title{Software Design and Evolution - Project}



\author{S\'{e}bastien Vaucher, 11-860-350\\Roger Stebler, 08-928-343}
% //////////////////////////////////////////////////


\begin{document}



\maketitle


% Add figures:
%\begin{figure}[t]
%%\begin{center}
%\quad\quad   \includegraphics[width=1\linewidth]{ass2}
%%\end{center}
%
%\label{fig:performance}
%\end{figure}

\section{General information}

\paragraph{Selection of the program}\mbox{}\vspace{10pt}\\
We first tried to use an .mse file provided by Pangea. The smallest of those projects contained almost 2000 model classes. So we decided to use a smaller project and generate an .mse file by ourself.


Finally we found the program JFLAP with less than 1000 classes. JFLAP is a software to experiment with finite automata, turing mashines and several types of grammars. \footnote{\url{http://www.jflap.org}}. So we created an .mse file for this program using VerveineJ and used that to do our static analysis.


\section{Static analysis}

\paragraph{Visualizing the call-graph}\mbox{}\vspace{10pt}\\
Since we are interested in finding long methods that are called often and are time consuming, we decided to only show the top 20 methods regarding LOC*NOIC in our visualization. Using the CHA algorithm we calculate the possible receivers of these methods and visualize the call-graph by showing the connections between the methods and the possible receiver classes.


The methods are displayed as red squares. Their sizes define the number of lines of code and their brightness show the number of incoming invocations. The classes are displayed as green circles. All the objects are aligned in two circles. The methods are laid out in the inner circle and the classes in the outer circle. Using this design we can clearly distinguish classes and methods.

\section{Dynamic analysis}


\paragraph{Off-the-shelf profiler}\mbox{}\vspace{10pt}\\
To profile JFLAP we tested \texttt{xprof} and \texttt{hprof}. \texttt{Xprof} is a really simple profiler, that does not provide much functionality. The data is split up for each thread and shows the number of invocations and the percentage of the runtime. \texttt{Hprof} did not run at all on our machines.

So we decided to use the shareware \texttt{JProfiler} to profile our program. It's a powerful tool that displays the collected data in real-time with a graphical representation. It can directly find bottlenecks in the code by ordering the methods using the total runtime (NOIC * T). It is also possible to collect information about the memory usage and some other Java-specific metrics.



\end{document}